\cleardoublepage
\chapter*{Notations}
\markboth{Notations}{Notations}
\addcontentsline{toc}{chapter}{Notations}

$\N$ denotes the set of natural integers and $\R$ the set of real numbers.

Vectors are written in bold letters, e.g. $\xx, \yy \in \R^p$.

Matrices are in capital bold letters, e.g. $\mX \in \R^{n \times p}$.

We may cast any vector $\xx \in \R^p$ to a column matrix $x \in \R^{p \times 1}$ with the same entry values.

We note $\odot$ the Hadamard (element-wise) product between two vectors or matrices.

$\0$ and $\1$ are the vectors whose all entries are $0$ and $1$ respectively, and whose size is inferred by the context.

For a vector $\vv$, we note $\norm{\vv}_0$ the number of non-zero entries of $\vv$ (that is, its cardinality).

The $\diag$ operator can be used in two contexts;
either to transform a vector into a diagonal matrix whose diagonal entries match the vector,
or to extract the diagonal of a matrix.
For example
\begin{equation*}
    \diag \begin{bmatrix}
         1\\
         2
    \end{bmatrix}
    =
    \left(\begin{array}{@{}cc@{}}
        1 & 0\\
        0 & 2
    \end{array}\right)
    ,\qquad
    \diag \left(\begin{array}{@{}cc@{}}
        1 & 2\\
        3 & 4
    \end{array}\right)
    =
    \begin{bmatrix}
         1\\
         4
    \end{bmatrix}
\end{equation*}

Given two matrices $A$ and $B$, we note $\big[ A, B \big]$ and $\big[ A; B \big]$ their horizontal and
vertical concatenation respectively (if dimensions match).

Functions applied to vectors are implicitly performed element-wise, unless otherwise specified.
