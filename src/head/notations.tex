\cleardoublepage
\chapter*{Notations}
\markboth{Notations}{Notations}
\addcontentsline{toc}{chapter}{Notations}

\begin{itemize}
    \item $\N$ denotes the set of natural integers and $\R$ the field of real numbers.
    \item Vectors are written in bold letters, e.g. $\xx,\, \yy \in \R^p$.
    \item For a vector $\vv$, we note $\norm{\vv}_0$ the number of non-zero entries of $\vv$ (that is, its cardinality).
    \item Matrices are written in capital letters, e.g. $X \in \R^{n \times p}$.
    \item Vectors $\xx \in \R^p$ are by default considered to be columns matrices in
        $\R^{p \times 1}$ with the same size and entry values.
    \item $\0$ and $\1$ are vectors (or sometimes matrices) whose all entries are $0$ and $1$ respectively,
        and whose sizes are inferred by the context.
    \item We note $\odot$ the Hadamard (element-wise) product between two vectors or matrices.
    \item Given two matrices $A$ and $B$, we note $\big[ A, B \big]$ and $\big[ A; B \big]$ their horizontal and
        vertical concatenation respectively (if dimensions match).
    \item The $\diag$ operator may be used in two contexts;
    either to transform a vector into a diagonal matrix whose diagonal entries match the vector,
    or to extract the diagonal of a matrix.
    For example,
    \begin{equation*}
        \diag \begin{bmatrix}
                  1\\
                  2
        \end{bmatrix}
        =
        \begin{bmatrix}
            1 & 0\\
            0 & 2
        \end{bmatrix}
        ,\qquad
        \diag \begin{bmatrix}
                  1 & 2\\
                  3 & 4
        \end{bmatrix}
        =
        \begin{bmatrix}
            1\\
            4
        \end{bmatrix}
    \end{equation*}
    \item The sets of $n \times n$ symmetric,
        symmetric positive semidefinite,
        and symmetric positive definite matrices
        are denoted by $S^n$, $S^n_+$ and $S^n_{++}$ respectively.
    \item Functions $\R \to \R$ applied to vectors or matrices are implicitly performed element-wise,
        unless otherwise specified.
    \item $\Eb{\cX}$ denotes the expectation of a random variable (or vector) $\cX$.
    \item Given two random variables (or vectors) $\cX,\, \cY$,
        we note $\cX \independent \cY$ the fact that they are independent.
    \item Given a set $\cS$, $\abs{\cS}$ and $\#\cS$ denote its cardinality (number of elements).
\end{itemize}
