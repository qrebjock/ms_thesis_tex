\cleardoublepage
%\chapter*{Abstract}
\markboth{Abstract}{Abstract}

\null\vspace{5cm}
\phantomsection
\begin{abstract}
\addcontentsline{toc}{chapter}{Abstract} % adds an entry to the table of contents
\centering
\begin{minipage}{\dimexpr\paperwidth-10cm}
      Feature selection is an active area of research in machine learning whose primary goal is to make models
      more interpretable by keeping most relevant covariates.
      Often, one is interested in selecting as many true positive features as possible
      (that is, maximizing true discoveries),
      while maintaining the false discovery rate under some threshold.
      In the past decades
      In this high-dimensional setting,
      that is when the number of features is larger than the number of samples,
      the problem of feature selection becomes particularly challenging.

      Recently, Barber-Candès introduced the knockoffs framework to control the false discovery rate
      when performing feature selection.
      The key idea is to build a knockoff feature for each original feature

      In this master thesis, we focus on the high-dimensional feature selection setting.
      Several bottleneck of the knockoff procedure are addressed.
      \vspace{2cm}
      \textbf{Keywords:} Feature selection, false discovery rate control, knockoff filter, high dimension
\end{minipage}
\end{abstract}
