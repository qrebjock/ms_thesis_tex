\cleardoublepage
%\chapter*{Abstract}
\markboth{Abstract}{Abstract}

\null\vspace{4.5cm}
\phantomsection
\begin{abstract}
\addcontentsline{toc}{chapter}{Abstract} % adds an entry to the table of contents
\centering
\begin{minipage}[b]{\dimexpr\paperwidth-10cm}
      High-dimensional feature selection has become a common task in many sciences over the past decades
      because of advancements in data acquisition.
      It is now possible to record dozens of thousands of features at low cost,
      and it is crucial to find out later which of them are of interest.
      In this master thesis, we focus on false discovery rate control,
      that is the problem of making as many relevant selections as possible
      while maintaining the number of false positives under a certain threshold.\\

      Recently~\cite{model_x_knockoffs} introduced the knockoffs framework to control the false discovery rate
      when performing feature selection.
      The key idea is to build a knockoff (fake) feature for each original feature, and compare their significance against
      a statistical model.
      Despite the theoretical possibility to employ this framework in high-dimension, several bottlenecks
      make it unemployable in practice.
      We take advantage of low-rank approximations to efficiently build knockoff features in high dimension,
      and show that a reasonable statistical power can be preserved by using accelerated significance statistics.
      We apply these techniques to genetic data and perform FDR control on
      $\approx 30\,000$ features in less than a minute on a standard workstation.\\\\\\

      \textbf{Keywords}: Feature selection, false discovery rate control, knockoff filter, high dimension
\end{minipage}
\end{abstract}
