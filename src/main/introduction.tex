\cleardoublepage
\chapter*{Introduction}
\markboth{Introduction}{Introduction}
\addcontentsline{toc}{chapter}{Introduction}

Feature selection
In the past two decades, acquiring data has become
Datasets end up with a number of features of several dozens of thousands.
Often, it is also easier to measure features than to label the sample with the right category.
Indeed, the latter often requires a human and is very costly.
in this setup, feature selection is particularly challenging as the number of observed samples might be much
smaller than the number of covariates.
It is for example the case of fMRI data and genomic data.

In this master thesis,
The first chapter introduces the concept of feature selection,
false discovery rate,
and the knockoff framework developed by Barber-Candès.
Chapter 2 introduces a sparse version of naive Bayes that scales linearly in the number of features against which
it is trained.
