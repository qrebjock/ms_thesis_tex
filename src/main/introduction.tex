\cleardoublepage
\chapter*{Introduction}
\markboth{Introduction}{Introduction}
\addcontentsline{toc}{chapter}{Introduction}

Feature selection is an active area of research in statistics and machine learning aiming at
making models more interpretable by identifying most relevant covariates explaining an observed phenomenon.
In the past two decades, acquiring data has become
Datasets end up with a number of features of several dozens of thousands.
Often, it is also easier to measure features than to label the sample with the right category.
Indeed, the latter often requires a human and is very costly.
in this setup, feature selection is particularly challenging as the number of observed samples might be much
smaller than the number of covariates.
It is for example the case of fMRI data and genomic data.

In this master thesis,
The first chapter introduces the concept of feature selection,
false discovery rate,
and the knockoff framework developed by Barber-Candès.
Chapter 2 introduces a sparse version of naive Bayes that scales linearly in the number of features against which
it is trained.

Feature selection is an active area of research in machine learning whose primary goal is to make models
more interpretable by keeping most relevant covariates.
Often, one is interested in selecting as many true positive features as possible
(that is, maximizing true discoveries),
while maintaining the false discovery rate under some threshold.
In the past decades
In this high-dimensional setting,
that is when the number of features is larger than the number of samples,
the problem of feature selection becomes particularly challenging.

Recently, Barber-Candès introduced the knockoffs framework to control the false discovery rate
when performing feature selection.
The key idea is to build a knockoff feature for each original feature

In this master thesis, we focus on the high-dimensional feature selection setting.
Several bottleneck of the knockoff procedure are addressed.
