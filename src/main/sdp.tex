\chapter{SDP}\label{ch:sdp}

In the case of gaussian knockoffs,
or more generally when we only impose $\left(\mX, \tilde{\mX}\right)_{\swap\left( S \right)}$ to have the same distribution
as $\left(\mX, \tilde{\mX}\right)$ for any subset $S \subseteq \left\{ 1, \dots, p \right\}$

\[
\left(\begin{array}{@{}cc@{}}
  \mathbf{\Sigma}
  & \mathbf{\Sigma} - \diag \bs\\
  \mathbf{\Sigma} - \diag \bs
  & \mathbf{\Sigma}
\end{array}\right)
\]

This matrix is positive semi definite if and only if $\diag \bs \preceq 2 \mSigma$.

\section{Equi-correlated knockoffs}\label{sec:equi}

A fast and easy way to find a suitable $\bs$ is to pick $s_j = \min(2\lambda_{\text{min}}(\mSigma), 1) \,\forall j$.

\section{SDP knockoffs}\label{sec:sdp}


It motivates us to solve the following optimization problem in order to

\begin{equation*}\label{eq:sdp}
\begin{aligned}
    & \underset{\bs}{\text{maximize}}
    & & \sum_{j = 1}^p s_j\\
    & \text{subject to}
    & & 0 \leq s_j \leq 1 \forall j\\
    & & \diag \bs \preceq 2 \mSigma
\end{aligned}
\end{equation*}

This problem is a structured semidefinite program (SDP) and can efficiently be solved for small values of p by interior point method for
example (\cite{interior_point_method_sdp}).
For larger values of $p$, it quickly becomes intractable and other methods must be considered.
Barber-Candès suggest to solve an approximated problem of~\ref{eq:sdp} in 2 steps that we describe below to reduce the computation time.

\paragraph*{Step 1.}
Pick an approximation $\mathbf{\Sigma_\text{approx}}$ of $\mSigma$ and solve
\[
    \text{minimize} \sum_{j = 1}^p (1 - \hat{s}_j)
    \quad\text{subject to}\quad
    0 \leq \hat{s}_j \leq 1, \diag{\hat{s}} \preceq 2\mathbf{\Sigma_\text{approx}}
\]
\paragraph*{Step 2.}
Solve
\[
    \text{maximize}\;\gamma
    \quad\text{subject to}\quad
    \diag\left\{\gamma\hat{s}\right\} \preceq 2\mathbf{\Sigma}
\]
Finally, pick $s = \gamma \hat{s}$.

Note that $\mathbf{\Sigma_\text{approx}} = \mathbf{I}$ and $\mathbf{\Sigma_\text{approx}} = \mathbf{\Sigma}$ yield
the last two solutions.
The step 2 can be solved very efficiently using bisection as it is a one-dimensional SDP\@.
Finding the right  $\mathbf{\Sigma_\text{approx}}$ is a compromise between computation time and power.
