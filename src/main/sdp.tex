\chapter{Efficient knockoffs construction}\label{ch:sdp}

As shown in Chapter~\ref{ch:fs}, in the case of gaussian knockoffs
$\tX$ can be sampled from a normal distribution $\cN\left( \bupsilon,\, \Upsilon \right)$ whose parameters
$\bupsilon,\, \Upsilon$ are formulated in~\ref{eq:conditional_gaussian_knockoffs}.
\begin{equation}\label{eq:conditional_gaussian_knockoffs}
    \tcX \mid \cX \sim \cN\left( \bupsilon, \Upsilon \right)
    ,\qquad\text{where}\quad
    \begin{cases*}
        \bupsilon = \cX - \cX\Sigma^{-1}\diag(\bs)\\
        \Upsilon = \diag(\bs)\left( 2\mI_{p \times p} - \Sigma^{-1}\diag(\bs) \right)
    \end{cases*}
\end{equation}
More generally, the same parameters are derived by only imposing
$\big[ X, \tX \big]_{\swap\left( S \right)}$ and $\big[ X, \tX \big]$
to have equal first two moments
for any subset
$S \subseteq \left\{ 1, \dots, p \right\}$,
rather than the same distribution.
These formulas are valid for any vector $\bs \in \R^p$ such that $\Omega$
is indeed a covariance matrix (semidefinite positive).
\begin{equation*}
    \Omega = \begin{bmatrix}
        \Sigma & \Sigma - \diag \bs\\
        \Sigma - \diag \bs & \Sigma
    \end{bmatrix}
    \succeq \0_{2p \times 2p}
\end{equation*}
As shown in proposition~\ref{prop:omega_psd},
this matrix is semidefinite positive if and only if $\0_{p \times p} \preceq \diag \bs \preceq 2 \Sigma$.
The first inequality clearly holds if all the entries of $\bs$ are positive.
It is however trickier to identify all vectors $\bs$ satisfying the second one.

In this chapter, we show why the choice of $\bs$ is important and how to efficiently find a good one
in the high dimensional setting.
In the remaining, we note $\Sigma$ the true covariance and $\hat{\Sigma}$ its estimation from the samples $X$
(be it empirical or derived from more sophisticated methods).

\section{Equi-correlated knockoffs}\label{sec:equi}

\subsubsection{A cheap solution}

For any psd matrix $A \in \R^p$, $A + \alpha I_{p \times p}$ has the same eigenvalues as $A$, but shifted by $\alpha$.
It gives a fast and simple way to find a feasible $\bs$.
\begin{equation}\label{eq:equi}
    s_j = \min(2\lambda_{\min}(\cove),\, 1)
    \quad\text{ for all } 0 \leq j \leq p
\end{equation}
The problem of finding the smallest eigenvalue $\lambda_{\min}(\cove)$ can be solved efficiently,
using for instance a singular value decomposition which runs in $\cO(p)$ steps~\cite{svd}.

\subsubsection{Why it is not desirable}

For large values of $p$,
the minimum eigenvalue of $\cove$ is likely to be very small,
unless $\cove$ has a substantially special structure.
Let us analyze briefly the covariance of $\big[ \cX; \tcX \big]$
to understand why it is unprofitable for the selection procedure.
If $\tcX$ is built according to~\ref{eq:conditional_gaussian_knockoffs}, it satisfies
\begin{equation*}
    \begin{cases}
        \cov(\tcX_j,\, \tcX_{j'}) = \cov(\cX_j,\, \cX_{j'}) \text{ for all } j,\,j'\\
        \cov(\cX_j,\, \tcX_{j'}) = \cov(\cX_j,\, \cX_{j'}) \text{ for all } j \neq j'\\
        \cov(\cX_j,\, \tcX_j) = \cov(\cX_j,\, \cX_j) - s_j \text{ for all } j
    \end{cases}
\end{equation*}
First, $\tcX$ has the same internal covariance has $\cX$,
and two distinct original and knockoff features have the same covariance
as the one of the two corresponding original features.
It makes the knockoff features sufficiently close to the original features
to fool the estimator computing the statistics.
However, an original feature $j$ and its corresponding knockoff have a covariance that is smaller the larger $s_j$ is.
If $s_j$ is close to $0$,
$\cX_j$ cannot be distinguished from $\tcX_j$ and the selection would suffer from a low power.

This fact is even more detrimental when $p$ is large as $\lambda_{\min}$ is likely to be very close to $0$.
It motivates us to maximize the sum of the entries of $\bs$ and it leads to solving a SDP\@.

\section{SDP knockoffs}\label{sec:sdp}

This observation motivates us to maximize the entries of $\bs$,
while maintaining the inequality $\diag\bs \preceq \tcovm$.
This can be formulated in the optimization problem~\ref{eq:sdp}.
\begin{equation}\label{eq:sdp}
    \underset{\bs \in \R^p}{\argmax}\;\,
    \sum_{j = 1}^p s_j
    \qquad
    \text{subject to } \begin{cases}
        s_j \geq 0\text{ for all } j\\
        \diag \bs \preceq \tcovm
    \end{cases}
\end{equation}
This problem is a structured semidefinite program (SDP) and can efficiently be solved for small values of
$p$ by interior point methods~\cite{interior_point_method_sdp} for example.
For larger values of $p$, even first order methods like
SCS~\cite{sdp_scs} or alternating direction~\cite{sdp_admm}
quickly become intractable.
In memory, it needs roughly $\cO(m^2)$ where $m$ is the number of constraints.
Even though the convergence speed depends a lot on $\covm$,
it experimentally appears that alternative methods have to be considered when $p > 1000$.
The Figure~\ref{fig:cvx_sdp_times} shows the convergence time of the solvers SCS and MOSEK as a function of $p$.
Theoretically, it takes $\cO(p^3)$ and it is clear that when $p$ is more than a few thousands.
Moreover, most solutions returned by these algorithms are actually infeasible because of numerical approximations.

\begin{figure}
    \centering
    \includegraphics[width=0.8\linewidth, height=0.5\linewidth]{figures/cvx_sdp_times.pdf}
    \caption{
        Time (in seconds) required to converge for SCS (blue) and MOSEK (orange) solvers
        as a function of $p$ (in log-log scale).
        Rapidly, MOSEK needs too much memory and can't be run for $p > 150$.
        Both match the theoretical $\cO(p^3)$ rate and SCS already needs 15 minutes for $p = 1000$.
        See Appendix~\ref{sec:cvx_times} for details on the data generation.
    }
    \label{fig:cvx_sdp_times}
\end{figure}

In order to reduce the computation time,
Barber-Candès suggest to solve an approximated problem of~\ref{eq:sdp} in 2 steps that we describe below.
\paragraph*{Step 1.}
Pick an approximation $\Sigma_\text{approx}$ of $\Sigma$ and solve
\begin{equation}\label{eq:sdp_approx}
    \underset{\hat{\bs} \in \R^p}{\argmax}\;\,
    \sum_{j = 1}^p \hat{s}_j
    \qquad
    \text{subject to } \begin{cases}
        \hat{s}_j \geq 0\text{ for all } j\\
        \diag \hat{\bs} \preceq \tcovmapprox
    \end{cases}
\end{equation}
\paragraph*{Step 2.}
Solve the one dimensional maximization
\begin{equation}\label{eq:sdp_approx_1d}
    \underset{\gamma \in \R}{\argmax}\;\,\gamma
    \quad\text{subject to}\quad
    \diag\left( \gamma \cdot \hat{\bs} \right) \preceq 2\Sigma
\end{equation}
Finally, pick $\bs = \gamma \cdot \hat{\bs}$.

\begin{remark}
    Note that the two extreme options $\covmapprox = \mI$ and $\covmapprox = \Sigma$ yield
    the same solutions as solving equi-knockoffs~\ref{eq:equi} and full SDP knockoffs~\ref{eq:sdp} respectively.
\end{remark}

Solving~\ref{eq:sdp_approx_1d} in step 2 can be done very efficiently using bisection as it is a one-dimensional SDP\@.
The optimization problem~\ref{eq:sdp_approx} is however the same as the one in~\ref{eq:sdp},
except for the approximation $\covmapprox$.
To speed up the computations,
$\covmapprox$ can be chosen to be a block-diagonal matrix of $k$ blocks.
The maximization then reduces to $k$ smaller SDPs for which the solutions can be found more efficiently.
These could even be distributed in several computation nodes.
Picking $\covmapprox$ is a compromise between available computation time and eventual power of the procedure.
There is a priori no ideal way to to it.
The block diagonal assumption depends on the order of the features.
They should therefore be reordered ahead of time,
typically by performing a clustering step to group very similar features.

\section{A coordinate descent approach}\label{sec:coordinate_descent}

\subsection{Notations and preliminaries}\label{subsec:notations_preliminaries}

Let $M \in \cS^p$.
Given two sets of indices $\cI,\,\cJ \in \fset$,
we note $M_{\cI,\,\cJ}$ the $\abs{\cI} \times \abs{\cJ}$ matrix obtained by keeping
the $\abs{\cI}$ rows and $\abs{\cJ}$ columns indexed by $\cI$ and $\cJ$ respectively.
By convenience, an integer $j$ denotes the set $\left\{ j \right\}$ and $j^c$ $\fset \setminus \left\{ j \right\}$
in the matrix subscripting context.
For example, if $M = \begin{bmatrix}
    1 & 2 & 3\\
    4 & 5 & 6\\
    7 & 8 & 9
\end{bmatrix}$,
then $M_{1^c, 1^c} = \begin{bmatrix}
    5 & 6\\
    8 & 9
\end{bmatrix}$
and $M_{1^c, 1} = \begin{bmatrix}
    4\\
    7
\end{bmatrix}$.

\bigbreak
Suppose $M$ is structured as
\begin{equation*}
    M = \begin{bmatrix}
        \xi & \yy^\top\\
        \yy & B
    \end{bmatrix}
\end{equation*}
where $B$ is symmetric.
\begin{lemma}
    Using Schur complements (see Appendix~\ref{sec:schur_complement}), M is psd if and only if
    \begin{equation*}
        B \succeq \0
        \quad\text{and}\quad
        \xi - \yy^\top B^{-1} \yy \geq 0
    \end{equation*}
\end{lemma}
In subscript notations, this is equivalent to $M_{1^c,\, 1^c} \succeq \0$ and
$M_{1,\, 1} - M_{1^c,\, 1}^\top M_{1^c,\, 1^c} M_{1^c,\, 1} \geq 0$.
This statement can be generalized as follows.
For any $j \in \fset$, M is psd if and only if both conditions are satisfied
\begin{equation}\label{eq:psd_schur_permutation}
    \begin{cases}
        M_{j^c,\, j^c} \succeq \0\\
        M_{j,\, j} - M_{j^c,\, j}^\top M_{j^c,\, j^c} M_{j^c,\, j} \geq 0
    \end{cases}
\end{equation}
\begin{proof}
    Let $j \in \fset$.
    Let $P$ be the permutation matrix changing $1 \leftrightarrow j$ and letting other columns unchanged.
    In two-line form, it is written
    \begin{equation*}
        P = \begin{pmatrix}
                1 & 2 & \dots & j - 1 & j & j + 1 & \dots & p\\
                j & 2 & \dots & j - 1 & 1 & j + 1 & \dots & p
        \end{pmatrix}
    \end{equation*}
    M is psd if and only if $P^\top M P$ is psd.
    $P^\top M P$ is $M$ with lines and columns $1$ and $j$ swapped.
    By applying Lemma~\ref{}, we get the result.
\end{proof}

\subsection{Coordinate ascent}\label{subsec:coordinate_ascent}

The previous observation will be useful to characterize the feasibility of a potential solution $\bs$ of the SDP\@.
Using~\ref{eq:psd_schur_permutation},
it appears that the feasibility of $\bs$ is equivalent to the three following conditions,
for any $j \in \fset$
\begin{equation}\label{eq:schur_sdp_constraints}
    \tcove \succeq \diag \bs \succeq \0
    \iff
    \begin{cases}
        \bs \geq \0\\
        \tcove_{j,\, j} - s_j
            - 4\cove_{j^c,\, j}^\top\Big( \tcove_{j^c,\, j^c} - \diag \bs_{j^c} \Big)^{-1}\cove_{j^c,\, j} \geq 0\\
        \tcove_{j^c,\, j^c} - \diag \bs_{j^c} \succeq \0
    \end{cases}
\end{equation}
This observation motivates the following approach, as described in~\cite{block_coordinate_sdp}.
Start with a feasible solution $\bs_0$, for example $\bs_0 = \0_p$.
At each iteration, only one coordinate $j$ of $\bs$ will be updated by optimizing
a sub-objective.
As only the coordinates $j$ changes, the constraints remain satisfied if we pick $s_j$ to satisfy
\begin{equation*}
    \begin{cases}
        s_j \geq \0\\
        s_j \leq \tcove_{j,\, j}
            - 4\cove_{j^c,\, j}^\top\Big( \tcove_{j^c,\, j^c} - \diag \bs_{j^c} \Big)^{-1}\cove_{j^c,\, j}
    \end{cases}
\end{equation*}

\subsection{Log-barrier}\label{subsec:log_barrier}



\section{Low-rank covariance matrix approximation}\label{sec:low_rank_sigma}

A~\cite{big_data_low_rank}

\subsection{Efficient SDP solving}\label{subsec:low_rank_sdp}

\begin{algorithm}
    \caption{Forecast shifting}\label{alg:forecast_shifting}
    \begin{algorithmic}[1]
        \State \textbf{Input:} Approximation $\cove = D + U \Lambda U^\top$
        \item[]
        \State $z'_t \gets \text{quantile}_\alpha(z_t) \; \forall t$ \algorithmiccomment{\emph{(i)}}
        \For{$t = n - 1,\dots, 1$}
        \State $z'_t := \max(z'_t, z'_{t + 1} - \hat{\rho})$
        \EndFor
        \item[]
        \State $\Delta_t \gets z'_t - z'_{t - 1}\; \forall t$ for $t = 2,\dots, n$ and $\Delta_1 = z'_1 - r_0$
        \algorithmiccomment{\emph{(ii)}}
        \State $\mathbf{r^+},\,\mathbf{r^-} \gets \max(\mathbf{\Delta}, 0),\,\min(\mathbf{\Delta}, 0)$
        \item[]
        \State Shift $\mathbf{r^+}$ and $\mathbf{r^-}$ in the past according to $\mathbb{E}$
        and $\mathbb{E}$ respectively and request them.
    \end{algorithmic}
\end{algorithm}

\subsection{Efficient low-rank sampling}\label{subsec:low_rank_sampling}

In this section, we detail briefly how to sample $\zz \sim \cN\left( \bupsilon,\, \Upsilon \right)$ once
a feasible $\bs$ is computed, and in the special case where $\cove = D + U\Lambda U^\top$.
A classical approach to sample from a multivariate normal distribution is to sample first a vector
$\tilde{\zz} \in \R^p$ from $\cN\left( 0,\, 1 \right)$,
and then pose $\zz = \bupsilon + L\tilde{\zz}$,
where $L$ is a lower Cholesky factorization of $\Upsilon$.
This uses the fact that if $\xx \sim \cN\big( \mu,\, \Sigma \big)$,
then
\begin{equation}\label{eq:affine_transformation}
    A\xx + \bb \sim \cN\big( A\bmu + \bb,\, A\Sigma A^\top \big)
\end{equation}
Note that this scheme works even if $L$ is not lower-triangular,
but only satisfies $LL^\top = \Upsilon$.

As $\Upsilon$ is a $p \times p$ matrix we may not want to store it in memory,
nor compute a Cholesky factorization (which takes $\cO\left( p^3 \right)$ steps).
We show here how to factorize $\Upsilon$ is a cheap way,
requiring only $\cO\left( k \cdot p \right)$ memory and $\cO\left( k\cdot p^2 \right)$ operations.
As shown in~\ref{subsubsec:gaussian_knockoffs}, $\Upsilon = 2S - S\cove^{-1}S$ where we note $S = \diag\bs$.
Using the low-rank structure $\cove = D + U\Lambda U^\top$,
the Sherman–Morrison–Woodbury formula (see Appendix~\ref{sec:sherman}) gives
\begin{align*}
    \cove^{-1} &= (D + UU^\top)^{-1}\\
    &= D^{-1} - D^{-1}U(I_k + U^\top D^{-1}U)^{-1}U^\top D^{-1}
\end{align*}
Note $L \in \R^{k \times k}$ the lower Cholesky factorization of
$(I_k + U^\top D^{-1}U)^{-1}$ (which can be computed efficiently if $k$ is small),
$V = SD^{-1}UL \in \R^{p \times k}$,
and $C = 2S - SD^{-1}S$.
Then the covariance reduces to
\begin{equation*}
    \Upsilon = C + VV^\top
\end{equation*}
where $C$ is diagonal and $VV^\top$ has rank at most $k$.
$C$ is not necessarily psd, which will force us to sample from a complex normal distribution.
Let $H$ be a complex square-root of $C$ (thus, potentially with imaginary numbers on the diagonal),
and $P = H^{-1}V \in \R^{p \times k}$.
Then,
\begin{equation*}
    \Upsilon = H \left( I_{p \times p} + PP^\top \right) H^\top
\end{equation*}
$\left( I_{p \times p} + PP^\top \right)$ can be factorized in the following way.
Note
\begin{equation*}
    W = \left(I_{k \times k} + \sqrt{I_{k \times k} + P^\top P}\right)^{-1} \in \R^{k \times k}
\end{equation*}
where the square-root is a $k \times k$ matrix root (which can again be computed efficiently if $k$ is small).
Then
\begin{equation*}
    I_{p \times p} + PP^\top = BB^\top
    ,\qquad
    \text{where}
    \quad
    B = I_{p \times p} + PWP^\top
\end{equation*}
Finally, we note $M = HB$ and we have that $\Upsilon = MM^\top$.
$B$ is $p \times p$ but we never have to fully evaluate it;
instead only $W$ and $P$ can be stored and matrix multiplications are then at most $p \times k$.

$M$ is a complex matrix so we cannot directly use the property~\ref{eq:affine_transformation}.
But if $\cX \sim \cN(\0,\, I)$ and $\cY = i\cX$,
then $\operatorname{Im}\left( M\cY \right) \sim \cN(\0,\, MM^\top = \Upsilon)$.
Using this scheme, with $p = 15\,000$ and $k = 100$ we can draw $n = 5\,000$ samples in $\approx 4$ seconds,
while the naive method not taking into account the low rank approximation would need $\approx 180$ seconds.
In comparison sampling $5000 \times 15\,000$ numbers from $\cN(0, 1)$ takes $\approx 2$ seconds.

\begin{figure}
    \centering
    \includegraphics[width=0.8\linewidth, height=0.5\linewidth]{figures/low_rank_times.pdf}
    \caption{
        Convergence times of the low-rank coordinate ascent algorithm~\ref{} as a function of $p$ and $k$.
        It confirms the theoretical $\cO(n_{\text{iters}} \cdot p \cdot k^3)$.
        In practice,
        the algorithm converges much faster to an appropriate solution when using a larger tolerance threshold.
        See Appendix~\ref{sec:coordinate_descent_data} for details regarding the random data generation.
    }
    \label{fig:low_rank_times}
\end{figure}