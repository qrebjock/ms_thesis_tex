\chapter{Sparse naive Bayes}\label{ch:snb}

Naive Bayes was first introduced in the early 60s by~\cite{original_naive_bayes} for text documents classification.
Despite its naive assumptions, naive Bayes remains an engaging model for large scale datasets because of its
low complexity.
An interesting propriety is that Naive Bayes can be trained online.
It can be particularly helpful if the dataset doesn't fit in memory.

\section{Reminders on vanilla naive Bayes}\label{sec:naive_bayes}

\subsection{General settings}\label{subsec:nb_general}

In this section, we recall briefly the naive Bayes model for binary classification.
Most key results can naturally be extended to the multiple classes setting.
Let $n$ and $p$ be two integers, $\mX \in \R^{n \times p}$ a feature matrix,
and $\yy \in \left\{ -1, 1 \right\}^n$ the associated target vector.
We will note the positive and the negative classes $\cC_+$ and $\cC_-$ respectively.
We wish to explain $\yy$ given $\mX$, that is, finding the posterior probabilities $\Pr(\cC_\pm \mid \xx)$
for any observed vector $\xx$.
To do so, we combine the use of Bayes rule and make the (very) naive assumption that
all the features are independent given the class, that is
\[
        \Pr\left( \xx \mid \cC_\pm \right) = \prod_{j = 1}^p \Pr(x_j \mid \cC_\pm)
        ,\qquad
        \Pr(\cC_\pm \mid \xx) = \frac{\Pr(\xx \mid \cC_\pm) \cdot \Pr(\cC_\pm)}{\Pr(\xx)}
\]
On the right hand side, the denominator $\Pr(\xx)$ does not depend on the class $\cC_\pm$ so we don't need to evaluate
it in order to perform classification.
Thus, we only need to estimate the probabilities $\Pr(\cC_\pm)$ and $\Pr(\xx \mid \cC_\pm)$.
The former are simply data averages, or the frequency of the positive and negative classes in the observed data.
As for the latter probabilities $\Pr\left( \xx \mid \cC_\pm \right) = \prod_{j = 1}^p \Pr(x_j \mid \cC_\pm)$,
they can be modeled by a plethora of distribution families, depending on the prior knowledge we have on the data,
and typically parametrized by some vector $\btheta \in \R^m$.
The probabilities are usually computed by maximizing the likelihood $\cL$,
or equivalently the log-likelihood $\lglh = \log\cL$, of the observed data.
\[
        \lglh\left( \btheta \right) = \sum_{i = 1}^n \Pr\left( \xx_i \mid y_i \,; \btheta \right)
\]

We present now 3 meaningful cases, where the prior distributions of $\Pr(\xx \mid \cC_\pm)$ are either
gaussian, bernoulli or multinomial.
Let $\cI_\pm = \left\{ i \in \pset \mid y_i = \cC_\pm \right\}$ be the sets of the positive and negative data points.
We also note for each class $\cC_\pm$ their cardinalities, empirical sums and averages respectively, as follows:
\begin{equation*}
        n^\pm = | \cI_\pm |,
        \qquad
        \ff^\pm = \sum_{i \in \cI_\pm} \xx_i,
        \qquad
        \bmu^\pm = \frac{\ff^\pm}{n^\pm}
\end{equation*}
In the following examples, the time complexity to train a naive Bayes classifier is $\cO(n \cdot p)$.
With a larger number of classes besides $\cC^+$ and $\cC^-$, say $k$, this complexity is multiplied by $k$.
In comparison, no closed-form solution exist for logistic regression~\cite{logistic_regression} based classifications.
A more costly gradient-descent based algorithm is executed to find the global optimum.

\subsection{Gaussian naive Bayes}\label{subsec:gnb}

In this case the observed data, conditioned on its label, is modeled by the gaussian distribution
$\cN( \bmu^\pm, \mSigma^\pm )$.
It is the most common case, to account for continuous data.
Note that the covariance $\mSigma^\pm \in \R^{p \times p}$ is diagonal because of the independence assumption that we made.
\[
        \Pr\left( \xx \mid C_\pm \right) =
        \frac{1}{\sqrt{(2\pi)^p\det(\mSigma_\pm)}}
        \exp\left( -\frac{1}{2}(\xx - \bmu_\pm)^\top\mSigma^{-1}(\xx - \bmu_\pm) \right)
\]
By denoting $\sigma_j = \Sigma_{j j}$, the log-likelihood can be written
\begin{equation*}
        \begin{split}
                \lglh\left( \bmu_+, \ssigma_+, \bmu_-, \ssigma_-\right ) &=
                        \sum_{j = 1}^p \bigg[
                                \sum_{i \in \cI_+}
                                        -\frac{1}{2} \log \left( 2\pi \right)
                                        -\log \sigma_j^+
                                        -\frac{(x_j - \mu^+_j)^2}{2{\sigma^+_j}^2}\\
                                &\qquad+ \sum_{i \in \cI_-}
                                        -\frac{1}{2} \log \left( 2\pi \right)
                                        -\log \sigma_j^-
                                        -\frac{(x_j - \mu^-_j)^2}{2{\sigma^-_j}^2}
                        \bigg]
        \end{split}
\end{equation*}
The minimizer of $\lglh$ admits a closed-form solution:
\begin{align*}
        \bmu^\pm = \frac{\ff^\pm}{n^\pm}
        ,\qquad
        \ssigma^\pm = \sqrt{\frac{1}{n^\pm} \sum_{i \in \cI^\pm} (\xx_i - \bmu^\pm)^2}
\end{align*}

\subsection{Bernoulli naive Bayes}\label{subsec:bnb}

Even though the Bernoulli case isn't very useful in practice, its solution is simple and elegant.
Assume that the design matrix is binary, that is $\mX \in \left\{ 0, 1 \right\}^{n \times p}$.
In order to model the conditional probabilities, we may assume the existence of $\btheta^+, \btheta^- \in \left( 0, 1 \right)^p$
such that for any data point $\xx \in \R^p$,
\[
        \Pr\left( x_j \mid \cC_\pm \right) = (\theta^\pm_j)^{x_j} \cdot (1 - \theta^\pm_j)^{1 - x_j}
\]
It yields that $\log \Pr\left( \xx \mid \cC_\pm \right) =
\xx^\top\log\btheta^\pm + (\1 - \xx)^\top\log\left( \1 - \btheta^\pm \right)$
and finally
\begin{align*}
        \lglh\left( \btheta^+, \btheta^- \right)
        &= \sum_{i \in \cI^+} \log \Pr\left( \xx_i \mid \cC_+ \right)
                + \sum_{i \in \cI^-} \log \Pr\left( \xx_i \mid \cC_- \right)\\
        \begin{split}
                &= \ff_+^\top\log\btheta^+ + (n_+\1 - \ff_+)^\top\log\left( \1 - \btheta^+ \right)\\
                &\qquad+ \ff_-^\top\log\btheta^- + (n_-\1 - \ff_-)^\top\log\left( \1 - \btheta^- \right)
        \end{split}
\end{align*}
The independence assumption makes the optimization problem decomposable across features;
it reduces to $p$ simple maximizations, and we find that $\theta_\pm^\star = \frac{f^\pm}{n^\pm}$.

\subsection{Multinomial naive Bayes}\label{subsec:mnb}

Multinomial naive Bayes is more general than the Bernoulli version
as we suppose that $\mX \in \N^{n \times p}$ is generated by the following underlying distribution,
parametrized by $\btheta^+, \btheta^- \in \left( 0, 1 \right)^p$ satisfying
$\1^\top\btheta^+ = \1^\top\btheta^- = 1$:
\[
        \Pr\left( \xx \mid \cC_\pm \right) =
                \frac{\big( \sum_{j = 1}^p x_j \big)!}
                        {\prod_{j = 1}^p x_j!} \cdot \prod_{j = 1}^p \theta_\pm^{x_j}
\]
Note that this model can still be used in the more general case where $\mX \in \R_+$, so it is not as restrictive as
it may appear at first sight.
The log probability is given by
\[
        \log\Pr\left(\xx \mid \cC_\pm\right) =
                \xx^\top\log\btheta_\pm + \log\frac{\big(\sum_{j = 1}^p x_j\big)!}{\prod_{j = 1}^p x_j!}
\]
and the log-likelihood reduces to
\[
        \lglh\left(\btheta^+, \btheta^-\right) = \ff_+^\top\log\btheta^+ + \ff_-^\top\log\btheta^-
\]
which is again decomposable across features.
It turns out that $\btheta_\pm^\star = \frac{\ff^\pm}{\1^\top\ff^\pm}$.

\subsection{Decision boundary}\label{subsec:nb_bound}

Given a new data point $\xx \in \R^p$, we wish to attach it the most probable label
$y \in \left\{ \cC^+,\, \cC^- \right\}$.
No matter what model parametrized by $\btheta$ was chosen,
\begin{align*}
        y &= \sign \log \frac{\Pr(\cC^+ \mid \xx \,;\,\btheta)}{\Pr(\cC^- \mid \xx \,;\,\btheta)}\\
        &= \sign\bigg[\log \frac{\Pr(\cC^+)}{\Pr(\cC^-)}
                + \log \frac{\Pr(\xx \mid \cC^+)}{\Pr(\xx \mid \cC^-)}\bigg]
\end{align*}
In the cases of \emph{bernoulli} and \emph{multinomial} naive Bayes,
there exist $v \in \R$ and $\ww \in \R^p$ such that $y = \sign(v + \ww^\top\xx)$.
In both cases, $v$ has the same value, and by noting $\ww_b$ and $\ww_m$ the weights for the bernoulli
and the multinomial case respectively, we have
\begin{equation*}
        v = \log\frac{\Pr(\cC^+)}{\Pr(\cC^-)}
        ,\quad
        \ww_b = \log(\btheta^+\odot(\1 - \btheta^-)) - \log(\btheta^-\odot(\1 - \btheta^+))
        ,\quad
        \ww_m = \log\btheta^+ - \log\btheta^-
\end{equation*}


\section{Sparse naive Bayes}\label{sec:snb}

\subsection{Problem statement}\label{subsec:snb_ps}

We detail here a sparse version of naive Bayes introduced by~\cite{sparse_naive_bayes} that can be employed to
perform feature selection.
Let $k \in \N$ be a desired level of sparsity.
We wish to train a naive Bayes classifier whose decision boundary depends on at most $k$ features.
For a vector $\vv \in \R^n$ we note $\norm{\vv}_0$ the number of non-zero entries
(the cardinality of the vector).
As shown in the previous section, for both bernoulli and multinomial naive Bayes,
there exist $v \in \R$ and $\ww \in \R^p$ such that the prediction $y(\xx)$ of a new data point $\xx\in\R^p$ is
$\sign(v + \ww^\top\xx)$.
Furthermore, the $j$th entry of $\ww$ is null if and only if $\btheta^+_j = \btheta^-_j$.
Note that the decision vector $w$ depends on the feature $p$ if and only if $\theta^+_p \neq \theta^-_p$
The problem can be phrased as follows:

\begin{equation*}
        \begin{aligned}
                & \underset{\btheta^+, \btheta^-}{\text{maximize}}
                & & \lglh_\text{b}\left( \btheta^+, \btheta^- \right)\\
                & \text{subject to}
                & & \norm{\btheta^+ - \btheta^-}_0 \leq k.
        \end{aligned}
        \qquad\qquad
        \begin{aligned}
                & \underset{\btheta^+, \btheta^-}{\text{maximize}}
                & & \lglh_\text{m}\left( \btheta^+, \btheta^- \right)\\
                & \text{subject to}
                & & \norm{\btheta^+ - \btheta^-}_0 \leq k\\
                & \text{ and }
                & & \1^\top\btheta^+ = \1^\top\btheta^- = 1.
        \end{aligned}
\end{equation*}

\subsection{Main results and resolution}\label{subsec:snb_th}

Surprisingly, despite the combinatorial constraints,
this optimization problem can be (approximately) solved very efficiently,
with an additional minor cost compared to vanilla naive Bayes.
For the Bernoulli case especially, an optimal solution can be computed in closed-form.
In the multinomial case, there is no closed-form solution, but a near-optimal one can easily be obtained.
\begin{theorem}
        Supposing that $X \in \left\{ 0, 1 \right\}^{n \times p}$ is modeled by the Bernoulli distribution,
        an exact solution to the problem can be computed.
        \begin{align*}
                \vv &= (\ff^+ + \ff^-) \odot \log\left( \frac{\ff^+ + \ff^-}{n} \right)
                        + (n\1 - \ff^+ - \ff^-) \odot \log\left( \1 - \frac{\ff^+ + \ff^-}{n} \right)\\
                \begin{split}
                        \ww &= \ff^+ \odot \log \frac{\ff^+}{n^+} + (n^+\1 - \ff^+) \odot \log (\1 - \frac{\ff^+}{n^+})\\
                        &\qquad + \ff^- \odot \log \frac{\ff^-}{n^-} + (n^-\1 - \ff^-) \odot \log (\1 - \frac{\ff^-}{n^-})
                \end{split}
        \end{align*}
        Let $\cI$ be the set of $p - k$ smallest elements of $\ww - \vv$, and let
        \begin{equation*}
                {\theta^+_\star}_j = {\theta^-_\star}_j = \frac{1}{n}(f_j^+ + f_j^-)
                \;\forall j \in \cI
                ,\qquad
                {\theta^\pm_\star}_j = \frac{f^\pm_j}{n^\pm}
                \;\forall j \notin \cI
        \end{equation*}
\end{theorem}

\begin{theorem}
Suppose that $\mX \in \R_+^{n \times p}$ is modeled by the multinomial distribution.
We define $\phi_k : \alpha \mapsto s_k(\hh(\alpha)) + C$ where $C$ is some constant,
$s_k$ is the sum of the k largest values of a vector, and
\begin{equation*}
        \begin{split}
                \hh(\alpha) &= \ff_+ \odot \log \ff_+ + \ff_- \odot \log \ff_-
                                - (\ff_+ + \ff_-) \odot \log (\ff_+ + \ff_-)\\
                        &\qquad - \ff_+ \log \alpha - \ff_- \log (1 - \alpha)
        \end{split}
\end{equation*}
Let $\alpha^\star$ be the minimizer of $\phi_k$, $\cI$ the set of the $p - k$ smallest entries of
$\hh(\alpha^\star)$, and $B_\pm = \sum_{j \notin \cI} f_i^\pm$.
A primal point can be reconstructed as follows:
\[
        {\theta^+_\star}_j = {\theta^-_\star}_j = \frac{f_j^+ + f_j^-}{\1^\top(\ff^+ + \ff^-)}
        \;\forall j \in \cI
        ,\quad
        {\theta^\pm_\star}_j = \frac{B_+ + B_-}{B_\pm}\frac{f^\pm_j}{\1^\top(\ff^+ + \ff^-)}
        \;\forall j \notin \cI
\]
Furthermore, it holds that $\psi(k - 4) \leq \phi(k) \leq \psi(k) \leq \phi(k + 4)$,
implying that the duality gap is small if $\psi(k) - \psi(k - 4)$ is small.
\end{theorem}
Experimentally, the duality gap quickly converges to $0$ as $k$ increases,
and the reconstructed primal point is near-optimal.
The time complexity of sparse naive Bayes is in both cases $\cO(n \cdot p + p \cdot \log k)$,
which is a minor additional cost compared to plain naive Bayes.

The authors experiment their method on several text datasets,
including \textsc{amzn}, \textsc{imdb}, \textsc{twtr}, \textsc{mpqa} and \textsc{sst2}.
They obtain test accuracies competitive with more costly methods like the Lasso,
while training their models several order of magnitude faster.

\section{Applications}\label{sec:nb_app}

The apparent low complexity of sparse naive Bayes compared to $\ell_1$-penalized methods such as
Lasso, logistic regression or SVMs makes is appealing for very large scale datasets.
We mention here a few applications to which we will come back later.

